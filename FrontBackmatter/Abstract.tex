%*******************************************************
% Abstract
%*******************************************************
%\renewcommand{\abstractname}{Abstract}
\pdfbookmark[1]{Abstract}{Abstract}
\begingroup
\let\clearpage\relax
\let\cleardoublepage\relax
\let\cleardoublepage\relax

\chapter*{Abstract}
Color fundus images of the retina play an important role in diagnosing and screening diabetic macular edema (DME). Development of DME can ultimately lead to blindness, if it is not diagnosed at its early stages. This fact makes diabetic subjects in need to visit an ophthalmologist at least once every year. In rural areas and underserved regions, content-based image retrieval (CBIR) might compensate the lack of expert ophthalmologists. In this work, we present a fully automated CBIR system that retrieves color fundus images according to their content (quantity and location) of exudates. The proposed system is based on localizing the fovea and dividing the macula into three concentric regions; fovea, parafovea and perifovea regions. In each region, the quantity of texture discontinuities is calculated and stacked to form a feature vector. The latter is used to describe the visual content of the image and used in the retrieval process. In this work, the retrieval criterion gives higher priority to lesions near the center of the fovea to represent the severity of the DME, where the severity of the DME depends on the distance between the lesion and the macula center. Nearest-ten retrieved images are used to evaluate the system performance. Retrieval precision of 79.2 \% has been achieved using the proposed CBIR system.

\vspace{1cm}
{\hspace{-0.6cm}Keywords: \textit{Some useful keywords}}   

\vfill

\endgroup			

\vfill